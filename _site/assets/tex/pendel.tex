%!TeX ts-program = xelatex
%!TeX encoding = utf-8 Unicode
\documentclass[a4paper]{article}

\usepackage[utf8]{inputenc}
\usepackage[T1]{fontenc}
\usepackage{textcomp}
\usepackage{amsmath,amsthm,amssymb}
\usepackage{import}
\usepackage{pdfpages}
\usepackage{xcolor}
\usepackage[norsk]{babel}

\usepackage{listings}
\lstset{
  basicstyle=\ttfamily,
  columns=fullflexible,
  breaklines=true,
  postbreak=\raisebox{0ex}[0ex][0ex]{\color{red}$\hookrightarrow$\space}
}

\title{Svingende pendelum}
\author{Simon Bakken-Jantasuk}

\begin{document}

\maketitle

\tableofcontents

\listoftables

\pagebreak

\section{Introduksjon} % (fold)
\label{sec:introduksjon}
\subsection{Hensikt} % (fold)
\label{sub:hensikt}
\begin{flushleft}
	\begin{enumerate}
		\item Å bestemme tyngdeakselersajonen, g, ved hjelp av en pendel. \\
		\item Å bruke feilforplantningsreglene til å finne måleusikkerheten i et beregnet resultat.\\
	\end{enumerate}
\end{flushleft}
\subsection{Oppsummering} % (fold)
\label{sub:oppsummering}
Vi har målt lengden til tråden, og regnet g ved at ser på $\phi$ som liten; sånn at vi kan se på pendelen som et harmonisk serie; som gir formelen (1) $g = \frac{4\pi^2}{T^2} l$ \\
Ved bruk av (1) så regner vi ut $g$-verdiene. Vi regner med usikkerhet av $g$, ved hensyn til usikkerhet av tid. \\

% subsection oppsummering (end)
% subsection hensikt (end)
% section introduksjon (end)


\textbf{Utstyr} 
\begin{enumerate}
	\item Kule
	\item Tråd
	\item Krok
	\item Linjal
	\item Stoppeklokke
\end{enumerate}

\section{Teori} % (fold)
\label{sec:teori}

Bevegelsen av pendelen har vertikal sirkelbevegelse. Dette vil si at summen av kreftene peker innover mot sentrum mens fartsretningen endrer. Det virker en gravitasjonel kraft $V_g$ og en snorkraft $F_s$ på kulen.  \\
For pendelen så er 
\[
\Sigma F = ma = mg\sin{\phi} 
\] 
Hvor 
\[
a = g\sin{\phi}
\] 
Vi vet at 
\[
a = -\omega^2x
\] 
Sånn at
\[
	g\sin{\phi} = -\omega^2x
\] 
Vi vet at
\[
x = l\phi 
\] 
Da må 
\[
	g\sin{\phi} = -\omega^2l\phi 
\] 
Vi vet at
\[
\omega = \frac{2\pi}{T} 
\] 
Da må
\[
\begin{gathered}
	g\sin{\phi } = -\frac{4\pi^2}{T^2} l\phi \\
	g = -\frac{4\pi^2l}{T^2} \frac{\phi}{\sin{\phi}} 
\end{gathered}
\] 
De vinkelene vi jobber med er svært små. Så vi kan godt se på dette som en harmonisk serie. 
\[
\begin{gathered}
	\lim_{\phi \to 0} g = \lim_{\phi \to 0} -\frac{4\pi^2l}{T^2} \frac{\phi}{\sin{\phi}}  \\
	g = -\frac{4\pi^2}{T^2} l
\end{gathered}
\] 


% section teori (end)

\section{Fremgangsmåte} % (fold)
\label{sec:fremgangsmåte}
\begin{enumerate}
	\item Fest tråden i kulen, og heng den opp slik at kulen får svinge fritt.
	\item Gjennomfør nødvendige målinger og før opp resultatene.
	\item Finn ut usikkerheten i de målte verdiene.
	\item Bruk de målte verdiene med usikkerhet til å beregne tyngdens akselerasjon med usikkerhet.
\end{enumerate}
\subsection{Målinger} % (fold)
\label{sub:målinger}
Lengden $l$ på tråden var $86\times 10^{-2}$ m
\begin{table}[htpb]
	\centering
	\caption{rundetid ($ T $), sekunder ($ s $)}
	\label{tab:label}
	\begin{tabular}{c c}
		$T_1$ & 1.90 s \\
		$T_2$ & 1.79 s \\
		$T_3$ & 1.89 s \\
		$T_4$ & 2.04 s \\
		$T_5$ & 2.13 s \\
		$T_6$ & 2.06 s \\
		$T_7$ & 1.94 s \\
		$T_8$ & 1.87 s \\
		$T_9$ & 1.91 s \\
		$T_{10}$ & 1.96 s \\
	\end{tabular}
\end{table}
% subsection målinger (end)
% section fremgangsmåte (end)
\newpage
\section{Resultat} % (fold)
\label{sec:resultat}
\subsection{Python} % (fold)
\label{sub:python}



\begin{lstlisting}[frame=single][language=Python]
from math import *
from pylab import *

rundeTid = [
    1.90,
    1.79,
    1.89,
    2.04,
    2.13,
    2.06,
    1.94,
    1.87,
    1.91,
    1.96
]

lengdeTraad = 86 * 10 ** -2

gjennomsnittVerdi = sum(rundeTid)/len(rundeTid)
variasjonsBredde = max(rundeTid) - min(rundeTid)
usikkerhet = variasjonsBredde/2

$$$
def g(i)
	Regner ut gravitasjons akselerasjonen basert ved hensyn til array rundeTid.
parameter
	(Number) variabel i som er lik variabel til array rundeTid
return 
	(Number) gravitasjons akselerasjonen
$$$

def g(i):
    return 4 * pi ** 2 / rundeTid[i] ** 2 * lengdeTraad

i = 0

while i < len(rundeTid):
    g(i)
    print(f"g_{i + 1}", g(i))
    i = i + 1

$$$
def beregnerUsikkerhet(usikkerhet)
	Beregner ut usikkerheten for gravitasjons akselerasjonen, ved hensyn til usikkerheten til tid.
parameter
	(Number) usikkerhet

return
	(Number) g regnet med usikkerhet 
$$$

def beregnerUsikkerhet(usikkerhet):
    return (4 * pi ** 2 / (gjennomsnittVerdi + usikkerhet) ** 2) * lengdeTraad

print("g med usikkerhet er mellom :", beregnerUsikkerhet(usikkerhet), beregnerUsikkerhet(-usikkerhet))
\end{lstlisting}
% subsection python (end)
\subsubsection{Output} % (fold)
\label{ssub:output}
\begin{lstlisting}[frame=single][language=Python]
Output:
	g_1 9.404830786633626
	g_2 10.596248288051994
	g_3 9.504616091304106
	g_4 8.158265844806659
	g_5 7.483400370241221
	g_6 8.000621910582382
	g_7 9.02100094052168
	g_8 9.709010592166601
	g_9 9.306608683903235
	g_10 8.837838176735577
	g med usikkerhet er mellom : 7.561296608239082 10.727691893446588
\end{lstlisting}


% subsubsection output (end)
% section resultat (end)
\section{Diskusjon} % (fold)
\label{sec:diskusjon}
En måte å redusere usikkerhet på, kunne ha vært å latt pendelen bevege av seg selv. Dog, så er resultat forventet. 
% section diskusjon (end)
\subsection{Konklusjon} % (fold)
\label{sub:konklusjon}
Vi har målt lengden til tråden, og regnet g ved at ser på $\phi$ som liten; sånn at vi kan se på pendelen som et harmonisk serie; som gir formelen (1) $g = \frac{4\pi^2}{T^2} l$ \\
Ved bruk av (1) så regner vi ut $g$-verdiene. Vi regner med usikkerhet av $g$, ved hensyn til usikkerhet av tid. \\

\begin{center}
	$g$ er $ca.$ mellom fra $7 m/s^2$ til $11 m /s^2$
\end{center}
% subsection konklusjon (end)
\end{document}
