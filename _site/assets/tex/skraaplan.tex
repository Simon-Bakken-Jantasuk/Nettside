%!TeX ts-program = xelatex
%!TeX encoding = utf-8 Unicode
\documentclass[a4paper]{article}

\usepackage[utf8]{inputenc}
\usepackage[T1]{fontenc}
\usepackage{textcomp}
\usepackage{amsmath,amsthm,amssymb}
\usepackage{import}
\usepackage{pdfpages}
\usepackage{xcolor}
\usepackage[norsk]{babel}

\usepackage{listings}
\lstset{
  basicstyle=\ttfamily,
  columns=fullflexible,
  breaklines=true,
  postbreak=\raisebox{0ex}[0ex][0ex]{\color{red}$\hookrightarrow$\space}
}

\title{Skråplan}
\author{Simon Bakken-Jantasuk}

\begin{document}

\maketitle

\tableofcontents

\listoftables

\pagebreak

\section{Introduksjon} % (fold)
\label{sec:introduksjon}
\subsection{Hensikt} % (fold)
\label{sub:hensikt}
\begin{flushleft}
	\begin{enumerate}
		\item Undersøke bevegelsen til en boks som beveger seg opp et skråplan
	\end{enumerate}
\end{flushleft}
\subsection{Oppsummering} % (fold)
Undersøkte bevegelsen til en boks, og har funnet ut at 
\label{sub:oppsummering}
% subsection oppsummering (end)
% subsection hensikt (end)
% section introduksjon (end)

\textbf{Utstyr} 
\begin{enumerate}
	\item Gradskive
	\item Boks
	\item Fjær
	\item Linjal
\end{enumerate}
\section{Teori} % (fold)
\textbf{a)}
\label{sec:teori}
\begin{figure}[h]
  \centering
  \includegraphics[angle=0,width=0.5\textwidth]{skråplan.png}
  \caption{I ro, så har vi normalkraft og gravitasjonskraft}
  \label{fig:skråplan.png} 
\end{figure} \\
\textbf{b)} \\
Vi vet at,	
\[
\mu = \frac{R}{N}
\]
Hvor $G_x = R$
\[
R = G\sin{\phi} \wedge N_y = G\cos{\phi}
\]
Det vil si,
\[
\mu = \frac{mg\sin{\phi}}{mg\cos{\phi}}
\]
Vi vet at,
\[
\tan{\phi} = \frac{\sin{\phi}}{\cos{\phi}}
\]
Da vet vi at,
\[
\mu = \tan{\phi}, Q.E.D
\] \\
\textbf{c)} \\
\begin{figure}[h]
  \centering
  \includegraphics[angle=0,width=0.5\textwidth]{skraaplanopp.png}
  \caption{I bevegelse opp: normalkraft, gravitasjonskraft og friksjonkraft}
  \label{fig:skraaplanopp.png} 
\end{figure} \\
Uttrykk for akselerasjonen blir,
\[
\begin{gathered}
	\Sigma F = ma  \\
	\vec{G_x} + \vec{R_x} = ma_x  \\
	a_x = \frac{\vec{G_x} + \vec{R_x}}{m} \\
\end{gathered}
\]
Vi vet at,
\[
G_x = mg\sin{\phi} \wedge R_x = \mu mg\cos{\phi} \\
\]
Da blir,
\[
\begin{gathered}
	a_x = \frac{mg \sin{\phi} + \mu mg \cos{\phi}}{m} \\
	a_x = g(\sin{\phi} + \mu \cos{\phi}) \\
\end{gathered}
\] \\
\textbf{d)} \\
For et legeme som beveger seg ned et skråplan,
\[
a_x = g(\sin{\phi} - \mu \cos{\phi})
\]
Her så blir $R$ negativ. \\
\textbf{e)} \\
Ifølge formelen så er $a_{opp} - a_{ned} = \mu 2g\cos{\phi}$, men dersom vi ikke bruker dette som utgangspunkt \\
Men $a_{opp}$ og $a_{ned}$, så får vi, \\
\[
\begin{gathered}
	g(\sin{\phi} + \mu\cos{\phi}) - g(\sin{\phi} - \mu\cos{\phi}) = 2\mu g \cos{\phi} \\
	\mu = \frac{a_{opp} - a_{ned}}{2g\cos{\phi}}, Q.E.D \\
\end{gathered}
\]
Dette er lov fordi vi har brukt $a_{opp}$ og $a_{ned}$ som utgangspunkt.
% section teori (end)

\section{Fremgangsmåte} % (fold)
\label{sec:fremgangsmåte}
\begin{enumerate}
	\item Velger en vinkel $\phi$ 
	\item Bruker fjæren for å skyte opp boksen
	\item Gjentar (2) flere ganger
	\item Tar målinger av (3)
\end{enumerate}
\subsection{Målinger} % (fold)
\label{sub:målinger}
Vinkel $\phi = 9^{\circ} \pm 0.1^{\circ}$
Lengden av boksen er $25 \times 10^{-2}$ cm
\begin{table}[htpb]
	\centering
	\caption{strekning (cm)}
	\label{tab:label}
	\begin{tabular}{ c  c }
		$S_1$ & $81$ \\
		$S_2$ & $84$ \\
		$S_3$ & $82$ \\
		$S_4$ & $81$ \\
		$S_5$ & $85$ \\
		$S_6$ & $84$ \\
		$S_7$ & $85$ \\
		$S_8$ & $84$ \\
		$S_9$ & $81$ \\
		$S_{10}$ & 85 \\
		$S_{11}$ & 85 \\
	\end{tabular}
\end{table}
% subsection målinger (end)
% section fremgangsmåte (end)
\newpage
\section{Resultat} % (fold)
\label{sec:resultat}
\subsection{Python: akselerasjon} % (fold)
\label{sub:python}
\begin{lstlisting}[frame=single][language=Python]
from math import *

g = 9.81 # m/s^2
friksjonsKonstant = 0.17 

def akselerasjon(vinkel, retning):
  if retning == "opp":
    return g * (sin(vinkel) + friksjonsKonstant * cos(vinkel))
  elif retning == "ned":
    return g * (sin(vinkel) -  friksjonsKonstant * cos(vinkel))
  else:
    raise ValueError("Retningene må være opp eller ned")

akselerasjonOpp = round(akselerasjon(radians(9), "opp"),20)
akselerasjonNed = round(akselerasjon(radians(9), "ned"),20)

print(akselerasjonOpp)
print(akselerasjonNed)

def friksjonTall(vinkel, akselerasjonOpp, akselerasjonNed):
    return tan(vinkel), (akselerasjonOpp - akselerasjonNed)/(2*g*cos(vinkel))

print(friksjonTall(radians(9),akselerasjonOpp, akselerasjonNed))
\end{lstlisting}
\begin{lstlisting}[frame=single][language=Python]
Output:
	3.1817899476551763 
	-0.11254574356584655
	(0.15838444032453627, 0.17)
\end{lstlisting}
% subsubsection output (end)
\subsection{Python: bevegelse} % (fold)
\label{sub:bevegelse}
\begin{lstlisting}[frame=single][language=Python]
from pylab import * 

g = 9.81  
angle = radians(9) 
mu = 0.17 

x = 0
v = 0


strekning = []
tid = []
t = 1 

while t <= 10:
    a = g * (sin(angle) + mu * cos(angle))
    v = v + a * t 
    x = v * t + 1/2 * a * t ** 2
    strekning.append(x)
    tid.append(t)
    t = t + 1

plot(tid, strekning)
show()
\end{lstlisting}
% subsection bevegelse (end)
% subsection python (end)
% section resultat (end)
\end{document}
