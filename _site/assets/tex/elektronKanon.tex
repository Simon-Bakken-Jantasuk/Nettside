%!TeX ts-program = xelatex
%!TeX encoding = utf-8 Unicode
\documentclass[a4paper]{article}

\usepackage[utf8]{inputenc}
\usepackage[T1]{fontenc}
\usepackage{textcomp}
\usepackage{amsmath,amsthm,amssymb}
\usepackage{import}
\usepackage{pdfpages}
\usepackage{xcolor}
\usepackage[norsk]{babel}

\usepackage{listings}
\lstset{
  basicstyle=\ttfamily,
  columns=fullflexible,
  breaklines=true,
  postbreak=\raisebox{0ex}[0ex][0ex]{\color{red}$\hookrightarrow$\space}
}

\title{Elektron Kanon}
\author{Simon Bakken-Jantasuk}

\begin{document}

\maketitle

\tableofcontents

\listoffigures

\listoftables

\pagebreak

\section{Introduksjon} % (fold)
\label{sec:introduksjon}
\subsection{Hensikt} % (fold)
\label{sub:hensikt}
\begin{flushleft}
	\begin{enumerate}
		\item Undersøke banen til elektroner som blir akselerert fra ro av en spenning $U_1$ og går mellom to ladde, parallelle plater.
	\end{enumerate}
\end{flushleft}
\subsection{Oppsummering} % (fold)
\label{sub:oppsummering}
\begin{flushleft}
	Vi har skjekket hvorledes elektronen påvirkes gjennom et homogent elektrisk felt.
\end{flushleft}
% subsection oppsummering (end)
% subsection hensikt (end)
% section introduksjon (end)

\textbf{Utstyr} 
\begin{enumerate}
	\item To spenning bokser
	\item  Katode
	\item Anode
\end{enumerate}

\section{Teori} % (fold)
\label{sec:teori}
\textbf{a)} \\
Vi vet at,
\[
W = qU
\]
Hvor,
\[
W = \frac{1}{2}m_ev^2 \wedge q = e \wedge U = U_1
\]
Hvor vi løser for $v$,
\[
v = \sqrt{\frac{2eU_1}{m_e}}, Q.E.D
\]
Dette er farten den får når den passerer anoden. \\
\textbf{b)} \\
Dersom plateavstanden $d = 5 \times 10^{-2}$ m og  $U_2 = 60 V$
Så kan vi finne ut det elektriske felt mellom to plater, ved,
\[
E = \frac{U_2}{d}
\]
Vi får,
\[
	E = \frac{60V}{5 \times 10^{-2} m} 
\]
Som blir,
\[
	E = 12 \times 10^{-2} \frac{V}{m}
\]
Vi vet at elektronet er negativt ladd, dersom den positive platen er på toppen. Så vil elektronene bli tiltrukket oppover. \\
\textbf{c)} \\
\[
	v = \sqrt{\frac{2eU_1}{m_e}} 
\]
Hvor,
\[
e = 1.60217646 \times 10^{-19}C \wedge m_e = 9.10938188 \times 10^{-31}kg
\]
\[
v \approx 45941095 \frac{m}{s} 
\]
Vi må ta hensyn til relativitetsteori, dersom $v$ er større eller er lik $1\%$ av lysets hastighet, og det er den. \\
\textbf{d)} \\

\begin{figure}[h]
  \centering
  \includegraphics[angle=0,width=0.5\textwidth]{bane1.png}
  \caption{Uten relativistisk effekt}
  \label{fig:bane1.png} 
\end{figure}

\begin{figure}[h]
  \centering
  \includegraphics[angle=0,width=0.5\textwidth]{bane2.png}
  \caption{Med relativistisk effekt}
  \label{fig:bane2.png} 
\end{figure}

\textbf{e)} \\
Vi vet
\[
	x = v_{0x}t \wedge y = \frac{1}{2}at^2 \wedge a_e = \frac{q_eE}{m}
\]
Vi løser for tiden $t$,
\[
	t = \frac{x}{v_{0x}}
\]
Vi får,
\[
y = \frac{1}{2}\frac{q_eEx^2}{mv^2}
\]
Vi vet at,
\[
E = \frac{U_2}{d}
\]
Det vil si at,
\[
y = \frac{1}{2}\frac{q_eUx^2}{dmv^2}
\]
Vi deler på y,
\[
y(2dm_ev^2) = q_eU_2x^2
\]
Vi deler på $2d$,
\[
m_ev^2 = \frac{1_eU_2x^2}{2dy}
\]
Vi vet at,
\[
p^2 = m^2v^2
\]
Da må vi gange med $m_e$, og vi får,
 \[
m_e^2v^2 = \frac{qeU_2x^2m_e}{2dy} = p^2, Q.E.D
\]
% section teori (end)

\section{Fremgangsmåte} % (fold)
\label{sec:fremgangsmåte}
\begin{enumerate}
	\item Noterer lengden langs x-aksen
	\item Vi bestemmer spenning for $U_1$
	\item Øker spenning for $U_2$
	\item Noterer lengden langs y-aksen
	\item Gjør steg (3) 3 ganger til, og (4) 3 ganger til
\end{enumerate}
\subsection{Målinger} % (fold)
\label{sub:målinger}
x-akse $10\times 10^{-2}$ m \\
spenningen $U_1 = 3 \times 10^{-3} V$
\begin{table}[htpb]
	\centering
	\caption{$U_2$}
	\label{tab:label}
	\begin{tabular}{c}
		1kV \\
		2kV \\
		3kV \\
	\end{tabular}
\end{table}
% subsection målinger (end)
% section fremgangsmåte (end)
\newpage
\section{Resultat} % (fold)
\label{sec:resultat}
\begin{table}[htpb]
	\centering
	\caption{oppgave f}
	\label{tab:label}
	\begin{tabular}{| c | c | c | c | c | c}
		$U_1$ & 1 & 1 & 1 & V\\
		$U_2$ & 1 & 2 & 3 & V\\
		$v$ & $0.04 \times 10^{9}$ & $0.04 \times 10^{9}$ &$0.04 \times 10^{9}$ & $\frac{m}{s}$ 
	\end{tabular}
\end{table}
\begin{figure}[h]
  \centering
  \includegraphics[angle=0,width=0.5\textwidth]{f.png}
  \caption{}
  \label{fig:f.png} 
\end{figure}
\begin{table}[htpb]
	\centering
	\caption{oppgave g}
	\label{tab:label}
	\begin{tabular}{| c | c | c | c |}
		$U_2$ & 1 & 2 & 3 \\
		$y$ & 1 & 2 & 3
	\end{tabular}
\end{table}
\section{Diskusjon} % (fold)
\label{sec:diskusjon}
\begin{flushleft}
	Det gir mening at vi har fått samme fart $v$. Dette er fordi akselerasjonen $a$ er konstant.
\end{flushleft}
Vi vet,
\[
a = \frac{F_e}{m_e}
\]
Vi vet at det er et homogent felt. Det vil si at kraften $F_e$ lik hele tiden. Dermed er akselerasjonen $a$ konstant. Det vil si at $\Sigma F = 0$. Farten $v$ konstant. 
% section diskusjon (end)
\subsection{Konklusjon} % (fold)
\begin{flushleft}
	Vi vet at feltet er homogent, og at elektriske kraften er lik hele veien. Dette vil si at akselerasjonen er konstant. Vi har konkludert at farten er dermed konstant hele veien. 
\end{flushleft}
% subsection konklusjon (end)
\end{document}
